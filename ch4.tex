\chapter{Tecnologie e strumenti}
\section{Introduzione}
In questo capitolo, verranno descritti i tool, le tecnologie e i framework utilizzati per lo sviluppo del progetto, proponendo le considerazioni che hanno portato alla loro scelta rispetto alle principali alternative, che verranno descritte anch'esse per fornire una panoramica della strumentazione software disponibile indipendentemente dal progetto oggetto di questa tesi.
Il focus sarà sulle tecnologie della suite Microsoft \emph{.NET}, per poi analizzare gli strumenti di terze parti e le tecnologie per la containerizzazione dei microservizi.

\section{Tecnologie .NET}
\subsection{Panoramica del framework}

\subsection{ASP.NET Core Web API}
\subsection{ASP.NET Core MVC}
\subsection{Entity Framework Core}
\subsection{Duende Identity Server}
\subsection{ASP.NET Core Authentication per OIDC e JWT}
\subsection{SignalR}
\section{Strumenti di terze parti}
\subsection{SQLite}
\subsection{Swagger}
\subsection{YARP}
\section{Containerizzazione}
\subsection{Docker}
\subsection{Docker Compose}
\section{Strumenti di sviluppo}
\subsection{Visual Studio}
\subsection{Git}
\section{Considerazioni finali}












% web API, web mvc, Duende identity server, entity framework core, sqlite, docker, docker compose, swagger, git, visual studio, YARP, aspnetcore authentication, jwt e openidconnect, signalR
%
%



%    Tradizionalmente, le applicazioni web prevedono la navigazione dell'utente in più pagine web, spesso generate dinamicamente dal server: tale è la filosofia su cui si incentra il modello ASP.NET Core MVC.
%     Con il passare degli anni, tuttavia, si è assistito a un'evoluzione verso approcci più moderni, come l'adozione di architetture a singola pagina (SPA) e l'utilizzo di framework JavaScript per la gestione della logica di interfaccia utente (e.g. React, Angular, Vue), che hanno reso possibile esperienze di navigazione più fluide, modulari e reattive, a scapito di complessità, manutenibilità, retrocompatibilità e performance.
%     PERCHè MVC E NON RAZOR PAGES O BLAZOR?

% PERCHE' NON MINIMAL API?