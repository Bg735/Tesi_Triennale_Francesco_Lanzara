% kestrel, nanoserver 1809
% windows usato per ragioni di diffusione nell'ambiente, anche se community meno ampia e immagini meno sviluppate rispetto a Linux

%Il progetto \emph{AuthServer} descrive il microservizio che fa da Identity Provider per l'applicativo. Utilizza il framework \emph{Duende Identity Server} per implementare e configurare con pochi metodi un servizio di autenticazione completo basato sul protocollo OpenID Connect. Per la gestione di utenti e ruoli si affida invece ad \emph{Identity} con \emph{EntityFramework Core}.
% \emph{EntityFramework Core} è l'\textbf{ORM} (\emph{Object-Relational Mapper}) ufficiale di Microsoft per \emph{.NET}. Esso consente di interagire con un database relazionale direttamente mediante oggetti \emph{C\#} (a cui le tabelle vengono appunto \emph{mappate}), astrarre le operazioni di lettura e scrittura dei dati e automatizzare la creazione e l'aggiornamento dello schema del database mediante il concetto di \emph{migration}, che consente di definire le modifiche allo schema come classi \emph{C\#} e applicarle al database in modo incrementale, mantenendo traccia delle modifiche già applicate. Per questo progetto infatti, la persistenza è stata gestita in maniera potente ed integrata senza dover scrivere una sola riga di codice \emph{SQL} o dover configurare manualmente il database.
% Il framework \emph{Identity} di ASP.NET Core, usufruisce della potenza di \emph{EntityFramework Core} (e la magia dello \emph{scaffolding}) per implementare un sistema di autenticazione e autorizzazione completo, con supporto per la gestione di utenti, ruoli, password e potenzialmente autenticazione a due fattori, e molto altro. Esso si integra perfettamente con \emph{Duende Identity Server}, che estende le funzionalità di autenticazione e autorizzazione per supportare i protocolli OpenID Connect e OAuth 2.0.

%    Tradizionalmente, le applicazioni web prevedono la navigazione dell'utente in più pagine web, spesso generate dinamicamente dal server: tale è la filosofia su cui si incentra il modello ASP.NET Core MVC.
%     Con il passare degli anni, tuttavia, si è assistito a un'evoluzione verso approcci più moderni, come l'adozione di architetture a singola pagina (SPA) e l'utilizzo di framework JavaScript per la gestione della logica di interfaccia utente (e.g. React, Angular, Vue), che hanno reso possibile esperienze di navigazione più fluide, modulari e reattive, a scapito di complessità, manutenibilità, retrocompatibilità e performance.
%     PERCHè MVC E NON RAZOR PAGES O BLAZOR?

% PERCHE' NON MINIMAL API?