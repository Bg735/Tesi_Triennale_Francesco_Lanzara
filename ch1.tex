\chapter{Contesto applicativo: il mondo della reportistica automatica}
\section{La reportistica aziendale automatizzata per favorire un modello decisionale data-driven}
Nel contesto dei \emph{financial services}, la capacità di organizzare i dati e tramutarli in insight strategici è centrale per ottimizzare i processi di enti ed imprese e garantirne il successo.
La reportistica aziendale, tutt'altro che mera espressione di adempimento burocratico, costituisce una colonna portante dei processi decisionali, fungendo da linfa vitale per la business intelligence, il controllo di gestione e la digitalizzazione integrata dei flussi operativi.
Attraverso di essa, le organizzazioni non solo attestano la propria performance, ma costruiscono le fondamenta per una governance informata, agile e proattiva.
Un sistema di reportistica configurabile e avanzato consente alle realtà orientate ai dati di raccogliere e visualizzare indicatori chiave di performance, dati economico-finanziari, operativi e qualitativi, rendere trasparenti le metriche utili al confronto tra reparti o rispetto agli obiettivi strategici, e condividere informazioni in tempi rapidi, anche in mobilità, tramite interfacce dedicate e report compilati automaticamente nei formati più diffusi.

Un prodotto di questa natura, soprattutto se orientato alla facilità di configurazione e  personalizzazione del servizio\footnotemark, consente un approccio computer-aided e data-driven al decision making, per una vasta gamma di utenti anche su piani gerarchici distinti: dagli operatori che necessitano di generare, scaricare e archiviare report sulle proprie attività, ricevendo avvisi tempestivi sull’esito delle elaborazioni; ai dirigenti e manager che impiegano dati aggregati e dashboard evolute per il controllo di gestione, l’analisi degli scostamenti e la pianificazione strategica; fino agli enti e alle aziende che intendono evolvere verso una gestione dei dati moderna, flessibile e basata sui dati, riducendo i tempi di produzione dei report e migliorando contestualmente la qualità delle decisioni.

\footnotetext{A tal fine, risulta imprescindibile segmentare le funzionalità per garantirne la modularità seguendo le esigenze di ciascun profilo utente, da quello operativo a quello dirigenziale. La scelta logica per ottenere questo risultato consiste nella progettazione di un'architettura basata su microservizi, di cui si discuterà in seguito.}

\section{Obiettivi del progetto}
Nel panorama appena descritto si concretizza l'obiettivo del tirocinio curriculare oggetto di questa trattazione: l’ideazione e la prototipizzazione di un'architettura software modulare e service-oriented, che esponga un servizio preesistente di generazione automatica di report rendendolo \emph{sicuro, scalabile, personalizzabile e facilmente integrabile} con tool di business intelligence o altri software gestionali preesistenti.
Come verrà approfondito nel capitolo 4, il progetto prevede l'implementazione di un'architettura a microservizi, isolati mediante l'impiego di container \emph{Docker} che garantiscono sia la scalabilità orizzontale sia la migrazione tra ambienti eterogenei.
Le funzionalità essenziali da implementare si concretizzano nell'utilizzo del motore di reportistica per realizzare un'interfaccia web, che sia accessibile sia mediante RESTful API, affinché il servizio sia agganciabile da altri applicativi per il settore dei financial services (e.g. tool di business intelligence quali PowerBI o Tableau), sia avvalendosi di un'interfaccia utente che renda immediati e intuitivi la generazione e la condivisione dei report, esportabili in formati diffusi quali Excel (.xlsm) e PDF.
Caratteristiche aggiuntive, come l'orchestrazione con \emph{\textbf{Docker Compose}}, l'introduzione \textbf{update asincroni} dell'interfaccia, l'implementazione di un sistema di \textbf{autenticazione} robusto e centralizzato, e l'utilizzo di un \textbf{gateway API} come intermediario tra client e microservizi completano il quadro di una soluzione moderna, scalabile e sicura.

\section{Tecnologie abilitanti}
Il framework adottato per la realizzazione dei servizi è \emph{Microsoft ASP.NET Core}, che consente di sviluppare applicazioni web e API in modo rapido ed efficiente, sfruttando le potenzialità del linguaggio C\# e dell'ecosistema \emph{.NET}.
Nello specifico, per la configurazione del container incaricato di esporre l'API RESTful si usufruisce del modello \textbf{ASP.NET Core Web API}, che rende agile e intuitiva la configurazione degli endpoint e la manutenzione della logica dell'API (documentazione dell'API esposta mediante il servizio \textbf{Swagger}); l'applicativo web con interfaccia è stato realizzato secondo il modello \textbf{ASP.NET Core MVC}, utile per realizzare applicativi tradizionali\footnotemark, separando distintamente la logica di presentazione dalla logica di business, e facilitando lo sviluppo e la manutenzione dell'interfaccia utente. All'interfaccia utente sono poi aggiunte feature dinamiche mediante la tecnologia \textbf{SignlR}, che consente l'invio di notifiche asincrone mantenendo comunicazioni in bilaterali real-time tra client e server.

Per quanto concerne la struttura organizzativa dell'applicativo, annoveriamo altre feature che forniscono al progetto un'architettura del tutto simile un approccio a microservizi: la soluzione è basata su container indipendenti, orchestrati mediante \textbf{Docker Compose}; un container dedicato centralizza la gestione dell'autenticazione usufruendo di \textbf{Duende Identity Server}, \textbf{EntityFramework Core} e \textbf{ASP.NET Core Identity e Authentication}. L'adozione di un gateway API per un reverse proxying potente e facilmente configurabile con \textbf{YARP}, per la gestione delle richieste e l'applicazione di policy di sicurezza, garantendo non solo modularità e manutenibilità evolute nel tempo, ma una navigazione tra i vari servizi fluida, eliminando la necessità di autenticazioni multiple nell'accedere a servizi distinti del sistema, la cui separazione interna non dovrebbe influire sull'esperienza dell'utente.

In definitiva, la soluzione proposta non solo è tecnicamente adeguata ai requisiti richiesti da organizzazioni data-intensive, ma si collocherebbe con piena efficacia in scenari reali, promuovendo usabilità, integrabilità ed evolvibilità. Essa incarna il passaggio dalla reportistica statica e reattiva a un ecosistema dinamico, interconnesso e capace di generare valore continuo, trasformando i dati nel più prezioso alleato per il governo dell’impresa.

\section{Struttura della tesi}
Il seguito della trattazione si articolerà nelle sezioni seguenti:

\begin{enumerate}
	\item \textbf{Contesto applicativo}, con cenni al settore dei \emph{financial services} e alle esigenze di digitalizzazione e automazione dei processi decisionali che ne caratterizzano l'evoluzione.
	\item \textbf{Accenni alle architetture distribuite orientate ai microservizi} e descrizione di concetti e principi propri degli applicativi \textbf{software web-based} impiegati contestualmente alla soluzione prodotta.
	\item \textbf{Panoramica sugli strumenti tecnologici} impiegati, con particolare riferimento alla suite \emph{ASP.NET Core}, framework centrale per la realizzazione della soluzione in esame.
	\item \textbf{Analisi del progetto}, con illustrazione della struttura progettuale, delle scelte implementative e delle prospettive di evoluzione verso scenari di deployment distribuito e multicanale.
	\item \textbf{Conclusione}
\end{enumerate}

\footnotetext{
    Per "\emph{tradizionali}" si intende applicazioni multi-pagina (\emph{MPA}) con rendering server-side, una delle filosofie originarie per la realizzazione di applicativi web. Nei capitoli successivi si argomenterà la scelta dell'utilizzo di tale modello rispetto a soluzioni meno mature, e.g. lo sviluppo di applicazioni singola-pagina (\emph{SPA}) con tecnologia \emph{Blazor}, sempre della suite ASP.NET.
}




---



\section{Introduzione}
Nel settore finanziario moderno i dati sono un vero asset strategico: la digitalizzazione spinge istituzioni e intermediari a interconnettere sistemi e ad automatizzare i processi. In questo nuovo ecosistema digitale è necessario abbattere i silos informativi e ripensare l’infrastruttura IT secondo paradigmi cloud-native.
La trasformazione digitale nei financial services richiede infrastrutture modernizzate e automazione dei processi interni. Ad esempio, grazie all’adozione di tecnologie come RPA e machine learning, l’efficienza operativa delle aziende può aumentare di oltre il 40\%, riducendo drasticamente errori e tempi di esecuzione. Al contempo le autorità di vigilanza italiane (Banca d’Italia, CONSOB, IVASS) impongono un livello crescente di adempimenti normativi: gli intermediari devono produrre report periodici accurati, tempestivi e conformi ai formati ufficiali per il monitoraggio prudenziale. Questa molteplicità di requisiti rende strategica l’automazione della reportistica.

Metoda Finance S.R.L., azienda ospitante del tirocinio, sviluppa soluzioni software per la gestione documentale e la compliance normativa. Ha creato un motore interno di reportistica in grado di generare automaticamente file Excel e PDF, con configurazioni delle fonti dati semplificate a livello di funzione. Tuttavia il sistema originario non disponeva di un’interfaccia esterna fruibile: un potenziale utente dovrebbe interagire direttamente con il codice del motore di generazione, risultando inutilizzabile per l'operatore casuale privo di competenze informatiche specifiche né una conoscenza almeno superficiale del sistema. Da qui l’esigenza di avvolgere il servizio di reportistica in un'architettura facilmente fruibile, modulare e scalabile. L’obiettivo è esporre le funzionalità tramite API RESTful, in modo che possano essere richiamate da interfacce web o strumenti BI esterni, e mediante una maschera human-friendly che renda l'accesso facile all'operatore senza necessitare competenze specifiche. L’intero sistema dovrà essere gestito come insieme di microservizi containerizzati, con un punto d'accesso univoco che fornisca un livello di sicurezza ed autenticazione per l'intera struttura. In questo modo si punta a rendere il reporting automatizzato facilmente configurabile, altamente disponibile e adattabile a future evoluzioni tecnologiche.

\section{Il problema affrontato}
La produzione manuale dei report finanziari presenta evidenti criticità. Richiede molte risorse di tempo e personale specializzato, espone a errori di trascrizione e incoerenze nei dati, e rende complessi aggiornamenti o riconfigurazioni rapide. Per quanto riguarda l’architettura software, un sistema monolitico ostacola l’estensione funzionale e la scalabilità: qualunque modifica (ad esempio per un nuovo tipo di report) può richiedere di ridistribuire l’intero applicativo, e i carichi elevati devono essere bilanciati sull’intera piattaforma. Un approccio più flessibile è dato dalle architetture a microservizi, che scompongono le funzionalità in servizi indipendenti. In queste architetture ogni microservizio svolge un compito specifico e comunica con gli altri attraverso API ben definite. Ciò garantisce maggiore manutenibilità e resilienza: per esempio, se un componente è sottoposto a un picco di traffico (ad esempio il servizio di calcolo dei dati), solo quel servizio verrà replicato, anziché scalare l’intera applicazione. I microservizi favoriscono inoltre un deployment incrementale: si possono aggiornare singole funzionalità senza interrompere il servizio complessivo.

Un altro aspetto critico è la coerenza dei dati in un sistema distribuito. Per gestire l’interazione tra microservizi in maniera affidabile spesso si ricorre a comunicazione asincrona tramite code o middleware di messaggistica (e.g. Kafka, RabbitMQ), che promuovono il disaccoppiamento tra componenti e permettono di assorbire picchi di carico senza perdita d'informazione, mentre le operazioni incentrate sulla manipolazione dei dati sono gestite mediante le cosiddette \emph{saghe}, che garantiscono una sicurezza transazionale quando l'elaborazione dei dati attraversa più nodi di una rete di microservizi. Detto ciò, in un’architettura containerizzata e orchestrata (tipicamente Kubernetes), ogni microservizio tende a mantenere il proprio schema dati, riducendo al minimo le dipendenze incrociate e consentendo una scalabilità orizzontale elastica che prescinde dall'host fisico di deploy del servizio.
La soluzione che si propone di realizzare è un prototipo su host singolo di un sistema di reportistica automatica orientato ai microservizi. Vincoli di tempo e risorse hanno imposto che l'effettiva simulazione del sistema fosse limitata al singolo nodo, rinunciando alla configurazione di un sistema di orchestrazione e all'utilizzo di broker di \emph{message queueing} per una comunicazione multi-host efficace, ma l'architettura è progettata per essere facilmente personalizzabile ed estendibile a un cluster distribuito.

\section{Confronto con lo stato dell’arte}
Esistono diverse soluzioni di mercato per la reportistica, ma ognuna presenta limiti in relazione a flessibilità e scalabilità. Ad esempio:

\begin{itemize}
    \item \textbf{ERP integrati} (come SAP o Oracle): offrono moduli dedicati al reporting, ma in genere sono sistemi piuttosto rigidi. Spesso richiedono personalizzazioni complesse per adattarsi a contesti specifici di business, rendendo onerosa la configurazione dei report rispetto alle esigenze variabili dell’azienda.
    \item \textbf{Strumenti di Business Intelligence commerciali} (Power BI, Tableau, Qlik): eccellono nella visualizzazione dei dati e nel data modeling, ma necessitano di infrastrutture complesse, competenze specifiche di progettazione report e spesso licenze software costose. Possono inoltre richiedere tempi
    lunghi di set-up per integrarsi con sistemi preesistenti.
    \item \textbf{Framework open-source di reporting e BI} (come Metabase, ReportServer, JasperReports): garantiscono maggiore flessibilità di personalizzazione e costi inferiori, ma impongono all’utente finale di possedere competenze tecniche elevate per installazione e integrazione. L’integrazione con sistemi legacy può risultare complicata e il supporto tecnico limitato.
\end{itemize}

Le architetture basate su microservizi dimostrano invece come sia possibile realizzare sistemi distribuiti, scalabili e manutenibili. I grandi player del settore fintech adottano microservizi perché consentono di innovare rapidamente, isolare i malfunzionamenti e reagire agilmente ai picchi di traffico. La soluzione proposta non vuole essere da meno: ci si premura dunque di realizzare non una semplice web application che esponga in rete il servizio di reportistica automatizzata, ma un'architettura sicura e scalabile, che coniughi una user experience fluida e non influenzata dall'organizzazione sottostante, a un'organizzazione modulare che fornisca performance elevate e facilità di manutenzione ed estensione.

Nella soluzione proposta infatti il motore di reportistica è implementato con tecnologie moderne, pur fornendo un'interfaccia basata su tecnologie consolidate e integrabili anche con strumenti terzi più maturi: ASP.NET Core garantisce elevate prestazioni e sicurezza, l'adozione di una filosofia MPA e di un'API RESTful garantisce interoperabilità con browser e servizi meno recenti, mentre SignalR abilita funzionalità real-time con push verso i client senza polling/refresh.
L’uso di container Docker fa sì che ogni microservizio sia leggero e indipendente: ogni servizio è isolato dal sistema ospite, e l'unica risorsa necessaria all'esecuzione è la presenza sul server di un host Docker. In questo modo, è possibile replicare, avviare o eliminare container in pochi secondi, consentendo una scalabilità orizzontale allineata alla domanda effettiva. Ad esempio, in caso di carichi di lavoro imprevisti il sistema può istanziare ulteriori container del servizio interessato in maniera trasparente senza impattare gli altri componenti.

Infine, viene adottato un API Gateway centralizzato, che funge da unico punto d'ingresso per il browser e per le richieste REST. Ciò permette di applicare in un solo punto regole di sicurezza, autenticazione (OpenID Connect per il browser, JWT per le API), e di aggregare e instradare le risposte in modo trasparente. In pratica, l’API Gateway astrae i dettagli interni dei microservizi, facilitando l’evoluzione indipendente di ogni componente.

Il prototipo descritto in questa tesi integra quindi tutte queste best practice: un’architettura a microservizi containerizzati pronti per un deploy in un contesto distribuito e orchestrato, un API Gateway per la sicurezza e il bilanciamento, ASP.NET Core e SignalR per il calcolo e il push dei dati, e uno strato di persistenza distribuita. In questo modo si ottiene un sistema di reportistica completamente automatizzato, in grado di generare documenti dinamicamente a partire da fonti dati configurabili e di esportarli in Excel o PDF senza intervento manuale. L’approccio risponde alle esigenze di efficienza, coerenza e adattabilità del contesto finanziario moderno, rispettando gli obblighi normativi e abilitando gli utenti a consultare i report da interfacce web o strumenti di BI esterni.
