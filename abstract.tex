\chapter*{Abstract}
In un'era di digitalizzazione dei processi bancari e finanziari, la possibilità di fruire istantaneamente di moli sempre maggiori di dati e di elaborarle in tempi brevi ha reso il \emph{Data-Driven Decision Making} un approccio vantaggioso e imprescindibile, risultando in un'esplosione del settore fintech.
Specialmente nel contesto dei servizi informatici destinati ai business, il mercato è invaso dalla richiesta di soluzioni personalizzate per incrementare sicurezza ed efficienza di aziende ed enti dediti a gestione dati, business intelligence, controllo di gestione e digitalizzazione dei processi.

Nel contesto di tali soluzioni si ergono i sistemi di reportistica automatizzata, il cui scopo è favorire analisi operative o attività di controllo strategico mediante l'elaborazione e l'aggregazione contestuale di dati in documenti, generati automaticamente, finalizzati a evidenziare aspetti cruciali per decisioni strategiche mirate.
Più in generale, un tale sistema eleva le capacità di enti e aziende di: raccogliere e visualizzare indicatori chiave di performance, dati economico-finanziari, operativi e qualitativi; rendere trasparenti i dati utili al confronto tra reparti o rispetto agli obiettivi strategici; condividere informazioni in tempi rapidi, anche in mobilità, tramite dashboard dedicate e report esportabili nei formati più diffusi; supportare, attraverso funzionalità di notifica real-time, la tempestività degli operatori e la reattività alle criticità o ai cambiamenti nei processi di business; attrarre la fiducia di investitori, partner e stakeholder tramite un sistema di controlli, logging e auditing centralizzato e sicuro.

Il presente elaborato illustra la progettazione e l’implementazione di una piattaforma web per l'accesso a un motore di generazione automatica di report finanziari, sviluppata nell’ambito del tirocinio curriculare svolto dallo scrivente presso \emph{Metoda Finance S.R.L.}, realtà di riferimento nel settore delle soluzioni informatiche per intermediari finanziari.
Tale soluzione si propone di distinguersi dalle soluzioni commerciali correnti mediante lo sviluppo di un’architettura orientata alla distribuzione, alla facilità di configurazione e di fruizione del servizio, concepita non solo per garantire scalabilità orizzontale e resilienza nei contesti distribuiti, ma altresì per rendere l’evoluzione funzionale un processo di semplice parametrizzazione anziché di invasiva modifica del codice sorgente. Ogni unità software costituisce un bounded context con contratti di interfaccia chiari (REST) e responsabilità univoche, e la configurazione dei report generabili si riduce alla semplice scelta di un parametro che fornisca l'algoritmo di generazione appropriato.
Dunque, la proposta converge ingegneristicamente verso un sistema estensibile e manutenibile: la struttura altamente modulare favorisce la configurazione dinamica di un ambiente personalizzato in cui diverse categorie di utenti, da operatori a dirigenti e manager, rispondono con zelo alle nuove esigenze di mercato, accedendo ovunque e senza sforzo a dati e \emph{insight} significativi per l'azienda.

L'architettura proposta adotta un \textbf{paradigma a microservizi}, in contrapposizione al tradizionale approccio monolitico, al fine di garantire scalabilità orizzontale, resilienza ai guasti e manutenibilità modulare. Tale approccio consente di suddividere il sistema in componenti autonomi che comunicano tramite protocolli stateless ed elaborano informazioni su database centralizzati: ciascun componente è responsabile di una specifica capacità funzionale, secondo i principi di responsabilità unica, disaccoppiamento e governance decentralizzata. Il sistema è orchestrato mediante \emph{Docker Compose}, favorendo la portabilità e l'isolamento degli ambienti, e integra un \textbf{API Gateway} per la gestione centralizzata delle richieste, il bilanciamento del carico e l’applicazione di policy di sicurezza.
Il sistema si presenta come \textbf{interfaccia web-based} per l’accesso a un motore di generazione di report preesistente, fruibile sia tramite \textbf{RESTful API} con ASP.NET Core WebAPI, utile per l’integrazione con strumenti di Business Intelligence, sia mediante un front-end interattivo con rendering dinamico usando ASP.NET Core MVC. Inoltre, la piattaforma implementa meccanismi di comunicazione real-time tramite SignalR, consentendo aggiornamenti asincroni dell'interfaccia utente.
L'applicativo sviluppato si orienta nel mercato delle soluzioni software per enti nel contesto dei financial services costituendo uno strumento indispensabile per creare valore attraverso la correlazione ed esplorazione di dati eterogenei, rafforzare la trasparenza e la compliance, favorire la rapidità d'intervento grazie a meccanismi di allerta in real-time e supportare l’evoluzione futura del sistema di gestione.