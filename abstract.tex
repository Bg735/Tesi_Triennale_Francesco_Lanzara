\chapter*{Abstract}
Nel contesto della digitalizzaziones dei processi bancari e finanziari, la reportistica automatizzata si configura come un elemento cardine per la governance del rischio, la compliance normativa e il supporto alle decisioni strategiche. Il presente elaborato illustra la progettazione e l’implementazione di una piattaforma software per la generazione di report finanziari, sviluppata nell’ambito del tirocinio curriculare svolto dallo scrivente presso \emph{Metoda Finance S.R.L.}, realtà di riferimento nel settore delle soluzioni informatiche per intermediari finanziari.

L’architettura proposta adotta un \textbf{paradigma a microservizi}, in contrapposizione al tradizionale approccio monolitico, al fine di garantire scalabilità orizzontale, resilienza ai guasti e manutenibilità modulare. Tale approccio consente di suddividere il sistema in componenti autonomi che comunicano tramite protocolli stateless ed elaborano informazioni su database centralizzati: ciascun componente è responsabile di una specifica capacità funzionale, secondo i principi di responsabilità unica, disaccoppiamento e governance decentralizzata. Il sistema è orchestrato mediante \emph{Docker Compose}, favorendo la portabilità e l'isolamento degli ambienti, e integra un \textbf{API Gateway} per la gestione centralizzata delle richieste, il bilanciamento del carico e l’applicazione di policy di sicurezza.
Il sistema si presenta come \textbf{interfaccia web-based} per l’accesso a un motore di generazione di report preesistente, fruibile sia tramite \textbf{RESTful API} con ASP.NET Core WebAPI, utile per l’integrazione con strumenti di Business Intelligence, sia mediante un front-end interattivo con rendering dinamico usando ASP.NET Core MVC. Inoltre, la piattaforma implementa meccanismi di comunicazione real-time tramite SignalR, consentendo aggiornamenti asincroni dell'interfaccia utente.
L'applicativo sviluppato si orienta nel mercato delle soluzioni software per enti nel contesto dei financial services costituendo uno strumento indispensabile per creare valore attraverso la correlazione ed esplorazione di dati eterogenei, rafforzare la trasparenza e la compliance, favorire la rapidità d'intervento grazie a meccanismi di allerta in real-time e supportare l’evoluzione futura del sistema di gestione.