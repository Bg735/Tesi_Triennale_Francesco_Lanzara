\chapter{Architetture a microservizi e servizi web evoluti}
\section{I limiti delle architetture monolitiche}
Le architetture a microservizi si propongono di estirpare alla radice problemi e limiti delle architetture monolitiche, che nel contesto dei sistemi distribuiti rappresentano la soluzione intuitiva e tradizionale. In questa sezione, daremo un breve sguardo alle difficoltà che emergono con l'approccio monolitico, per poi introdurre il concetto di microservizi e come questi affrontino tali sfide.
Un applicativo si definisce \emph{monolitico} quando:

\begin{itemize}
  \item Un' unica codebase racchiude l'intera logica di business, che è al più suddivisa in moduli. Molti team di sviluppatori con competenze disparate devono collaborare su un unico progetto, integrare grandi quantità di modifiche dipendenti da più moduli e dispiegarle contemporaneamente sui server di produzione;
  \item Ogni modulo elabora l'interità della logica per il dominio a cui appartiene, e il suo \emph{lifecycle}, ossia l'estensione nel tempo dei periodi di attività e stallo del servizio, è dipendente da ogni altro modulo: se un modulo va "down" (per manutenzione o guasto), l'intero applicativo deve smettere di funzionare;
  \item I dati di tutti i domini sono gestiti in un unico database centralizzato, accessibile da ogni modulo. Questo è spesso un voluminoso database relazionale che consente di utilizzare facilmente \emph{join} per operazioni inter-modulari, ma può altrettanto facilmente diventare un collo di bottiglia per le performance e la scalabilità dell'applicativo.
  \item L'applicativo è distribuito come un unico eseguibile, quindi il carico di lavoro assegnabile ad ogni modulo è standardizzato per ogni singola istanza di esecuzione.
  Ipotizziamo un applicativo con moduli \textbf{A}, \textbf{B} e \textbf{C}, in cui le richieste al solo modulo \textbf{A} superano la sua capacità elaborativa. Per far fronte al problema, sarebbe necessario avviare una seconda istanza dell'applicativo, compresa di moduli \textbf{B} e \textbf{C} (non saturi), causando una dispersione evidente di risorse computazionali e di memoria.
\end{itemize}
Per quanto possa sembrare più \emph{semplice} e \emph{intuitivo} iniziare un progetto come soluzione monolitica, i bassi livelli di scalabilità ed evolvibilità della soluzione rendono facile comprendere l'adozione progressivamente maggiore di un approccio basato su microservizi per la progettazione di software distribuiti agili.

\section{I microservizi: ultima frontiera dei sistemi distribuiti}
Il concetto di architettura a microservizi non nasce come una realtà isolata, ma si colloca nel contesto delle \emph{Service Oriented Architecture}s (\textbf{SOA}). Questo termine, che nel tempo ha assunto connotazioni diverse per persone diverse, può essere definito a grandi linee come la pratica di decomporre un'applicazione in più servizi rappresentanti sottosistemi logicamente indipendenti, in una separazione orizzontale (componenti di uguale importanza ma che trattano aspetti diversi della logica di business) o verticale (divisione della logica in tier gerarchici).
Tuttavia, le architetture a microservizi si distanziano dalle best practices delle SOA, che spesso sono invece considerati anti-pattern per un sistema a microservizi; alcuni esponenti sostengono infatti come i microservizi siano "\emph{Se i SOA fossero ben fatti}" \cite[25]{.NET_Microservices}.

Più in generale, i microservizi specializzano il concetto più generico di SOA, introducendo tecniche e requisiti più restrittivi e che mirano a ottenere un risultato specifico: quello di definire un applicativo sulla base di servizi della dimensione minima necessaria ad operare su un contesto ben delineato e con un basso accoppiamento tra di essi, favorendo un deploy indipendente e una forte scalabilità orizzontale. In sostanza, i microservizi "\emph{forniscono agilità a lungo termine}" \cite[25]{.NET_Microservices}.

Per chiarire i problemi che un'architettura a microservizi si propone di risolvere, in opposizione al tradizionale approccio monolitico (in cui un unico eseguibile racchiude interamente la logica di business), ritengo utile elencare alcune delle caratteristiche principali di questo paradigma architetturale.

\subsection{Incapsulamento e indipendenza}
I microservizi sono ideati per essere autonomi e indipendenti, così da poter essere sviluppati, testati e distribuiti in maniera isolata. Ogni microservizio mantiene i propri runtime e dipendenze isolati dall'host su cui è eseguito, riducendo al minimo indispensabile le interazioni con altri servizi: questo approccio consente una maggiore flessibilità e agilità nello sviluppo e nella distribuzione delle applicazioni. Il ciclo di vita di un microservizio diventa così indipendente da quello degli altri elementi dell'ecosistema con cui interagisce.

Una conseguenza della separazione concettuale del singolo microservizio dall'architettura complessiva è che anche il modello concettuale diventa specifico e indipendente. In un'applicazione monolitica, il \emph{modello} di ogni dominio è sovrapposto agli altri: come risultato, spesso si lavora con singole entità che ricoprono ruoli diversi nel contesto di servizi differenti, generando ambiguità semantica e confusione nella gestione di database centralizzati, lo standard in un approccio monolitico.
Viceversa, in un'architettura a microservizi, ogni microservizio ricopre esattamente un dominio indipendente (in maniera non dissimile dai \emph{bounded context} in \emph{Domain-Driven Design}): in questo modo, può definire un \emph{modello concettuale} dalla struttura snella, efficiente e context-coherent, e gestire un eventuale \emph{state} su database per-service che riflettano tale struttura.
Tale soluzione presenta evidenti benefici: modificare l'implementazione (lo schema o addirittura il DBMS) non impatterà gli altri servizi che accedono ai dati, mentre la creazione di un singolo punto di accesso ai dati riduce la possibilità di incoerenza nei dati e aiuta a prevenire e individuare bug.
La separazione in database distinti consente inoltre di scegliere, per ogni contesto, \emph{tipi} diversi di database, magari coniugando soluzioni SQL e NoSQL: tale approccio è detto \emph{persistenza poliglotta}\cite[29]{.NET_Microservices}.

In sintesi, l'incapsulamento di logica e dati di un microservizio consente di sviluppare e aggiornare il servizio in maniera indipendente, restringendo la coordinazione con team assegnati ad altri domini alla sola progettazione delle API pubbliche.

\subsection{Smart Endpoints e Dumb Pipes per una comunicazione scalabile}
Spesso il design di SOA di software complessi si riduce a un collage di monoliti, la cui logica di comunicazione viene centralizzata mediante infrastrutture software (i cosiddetti \emph{service bus}). Tale pratica comune ha come ineluttabile conseguenza quella di accoppiare i servizi con l'infrastruttura, riducendo drasticamente l'indipendenza dei team di sviluppo che si ritrovano a far defluire parte della logica nella "terra di mezzo" del middleware di comunicazione, pronta alla "contaminazione" da parte di altri domini.
In opposizione a questa pratica, i microservizi adottano una filosofia di comunicazione \emph{smart endpoints, dumb pipes}\cite[19]{Thesis_microservices}, abolendo l'uso di di middleware complesso e preferendo invece sistemi di comunicazione leggeri e \emph{data-agnostic}: l'unico scopo dell'infrastruttura è recapitare i messaggi agli endpoint, evitando \emph{leak} del contenuto informativo e della logica di business al di fuori dei microservizi ai capi della trasmissione.

Le tipologie di comunicazione adottate nel contesto di un'architettura a microservizi si scandiscono in base a due criteri di selezione\cite[19]{Thesis_microservices}:
\begin{itemize}
  \item Comunicazione sincrona o asincrona: determina rispettivamente se il mittente attende una risposta dal destinatario prima di proseguire l'elaborazione, o se può continuare a operare indipendentemente dalla ricezione del messaggio. Una soluzione sincrona diffusa è l'utilizzo del protocollo \textbf{HTTP} per esporre \textbf{API RESTful}, garantendo una comunicazione chiara e \emph{stateless}. Vari pattern di messaggistica asincrona sono invece implementabili mediante il protocollo \textbf{AMQP};
  \item Comunicazione single-receiver o multi-receiver: stabilisce se il messaggio viene inviata a un singolo interlocutore (il fornitore del servizio richiesto) o in generale a più destinatari (tipici di questa categoria sono protocolli di tipo \emph{publish/subscribe}, fondamentali nella progettazione di sistemi event-driven).
\end{itemize}

\section{I trade-off della suddivisione in microservizi}
Nelle ultime pagine abbiamo esplorato alcuni problemi delle architetture monolitiche/SOA e come i microservizi si propongono di risolverli. Tuttavia, la scelta di separare un sistema software unico in applicativi indipendenti introduce alcune complicazioni, prevalentemente legate alla necessità di coordinare operazioni su più microservizi, richiedendo uno scambio di informazioni che sarebbe invece triviale tra moduli dello stesso processo.
Discutiamo brevemente in questa sezione le colonne portanti del layer di complessità derivato dall'introduzione dei microservizi.

\subsection{Resilienza distribuita delle operazioni}
In sistemi monolitici, invocare un'operazione su un'entità dello stesso processo è banale, grazie a funzionalità integrate nei linguaggi di programmazione e nei framework consolidati che consentono una gestione generalmente intuitiva del \emph{control flow}.
Invece, in un'architettura a microservizi distribuita (e magari orchestrata), una richiesta potrebbe essere smistata sullo stesso host come in un altro server a distanza variabile, introducendo una latenza non stimabile a priori e in generale la possibilità di estendere la comunicazione a una rete inaffidabile.

Se un applicativo monolitico può introdurre facilmente indicatori di progresso e lanciare eccezioni o errori in caso di fallimento di un'operazione, l'apertura dei microservizi a una comunicazione sincrona in rete fa sì che, finché non riceve una risposta, a un mittente non è dato sapere se l'operazione sia in corso, sia fallita e nemmeno se la richiesta sia arrivata al destinatario.

Una soluzione comune è l'introduzione di un timeout, che consente al mittente di interrompere l'attesa di una risposta e gestire il fallimento dell'operazione in modo efficace.
A tale soluzione si aggiungono solitamente alcune "strategie"\cite{Thesis_microservices} che consentono di reagire dinamicamente allo stato della rete per limitare i disagi causati da fallimenti "localizzati" delle richieste.
Tra queste strategie si annoverano: \textbf{(i)} \textbf{Retry con Backoff esponenziale e Jitter}: Quando una chiamata fallisce, il \emph{caller} ritenta la chiamata (con un limite di tentativi).
Per non sovraccaricare il \emph{callee}, i \emph{retry} sono ritardati con andamento esponenziale, su cui è introdotto un \emph{jitter}, cioè una deviazione dal tempo calcolato per evitare di inondare il \emph{callee} di richieste contemporanee. (\textbf{ii}) \textbf{Circuit Breaker}: le richieste a un microservizio fornitore falliscono automaticamente dopo un numero prestabilito di fallimenti. Lo scopo è quello di evitare attese che con ogni probabilità precedono un timeout. (\textbf{iii}) \textbf{Bulkhead}: Le risorse per l'accesso concorrente sono suddivise per dipendenza, così che un \emph{callee} guasto saturi solo una porzione dei worker asincroni. (\textbf{iiii}) \textbf{Fallback/Caching} In applicazioni in cui ricevere l'ultima versione dei dati non sempre è cruciale, è possibile introdurre strutture che forniscano risultati alternativi o meno recenti in caso di fallimento della chiamata.

\subsection{Integrità distribuita dei dati}
Pur assumendo una comunicazione infallibile tra microservizi, i meccanismi forniti dai DBMS per garantire l'integrità dei dati, prime tra tutti le \textbf{transazioni}, sono indissolubilmente connessi al concetto di operazioni sui dati come elaborazioni compatte e indipendenti, il cui esito non può essere certamente determinato in un processo multi-step su una rete inaffidabile.
Con il termine \emph{transazione} ci si riferisce ad una sequenza di operazioni effettuate su una base dati che garantisce il rispetto delle proprietà ACID:
\begin{itemize}
  \item Atomicity: le operazioni eseguite sono indivisibili: il risultato della transazione viene applicato (\emph{commit}) solo se tutte le operazioni che la compongono hanno avuto successo, altrimenti i loro effetti sono annullati (\emph{rollback});
  \item Consistency: iniziando in uno stato coerente dei dati, una transazione può solo terminare nello stato precedente o un nuovo stato coerente.
  \item Isolation: ogni transazione non deve interferire con altre transazioni in esecuzione.
  \item Durability: alla fine di una transazione le modifiche sono salvate in maniera persistente, puntando a garantire l'integrità dei dati anche in caso di guasti o danni fisici al sistema.
\end{itemize}

In un contesto distribuito, non è semplice garantire le proprietà \emph{ACID} di transazioni multi-servizio, poiché ognuno di essi gestisce i propri dati in maniera autonoma e le comunicazioni tra di essi possono fallire o essere ritardate. E' necessario introdurre meccanismi di \emph{error safety} per garantire che, in caso di fallimento di una parte del sistema, l'integrità dei dati sia preservata e le operazioni possano essere ripristinate ad uno stato coerente.
La soluzione è una "sovrastruttura" che diriga le transazioni per garantire uno stato coerente dei dati: si introduce così il concetto di \emph{saga}.

\subsubsection{Le saghe per dirigere le transazioni distribuite}
L'obiettivo del pattern \emph{saga} è quello di garantire la consistenza logica tra i database di più servizi, in linea con il principio di \emph{eventual consistency} (coerenza finale eventuale) proprio della programmazione distribuita: ogni operazione che coinvolga l'update di più database \emph{prima o poi} deve terminare garantendo uno stato coerente.

Nel concreto, come spiegato nelle guide Microsoft\cite{Microsoft_Saga_Pattern}, una saga consiste in una sequenza di transazioni locali, in cui ogni (micro)servizio esegue l’operazione e avvia il passaggio successivo tramite eventi o messaggi (in alcuni casi, lo stato della saga può essere dedotto dal contesto, come nel caso di timeout nell'attesa di una transazione). Ogni transazione locale esegue il proprio \emph{commit} e con una notifica richiede l'inizio della fase successiva. Se uno dei passaggi fallisce, la saga prevede l'esecuzione di transazioni compensative per invertire le modifiche apportate agli step precedenti.
I due approcci principali per coordinare le saghe sono \emph{coreografia} e \emph{orchestrazione}, tuttavia l'uso dell'orchestrazione è scoraggiato per coerenza con il principio di decentralizzazione proposto con la filosofia \emph{smart endpoints, dumb pipes}.

L'orchestrazione prevede un componente centrale (\emph{orchestrator}) che riceve gli eventi di avvio della saga e invia ordini ai servizi coinvolti. Tale soluzione evita dipendenze cicliche tra servizi e semplifica la logica di rollback complessivo, tuttavia l'orchestratore costituisce un \emph{single point of failure}, non sempre scala bene, e l'approccio iterativo del flusso di controllo (opposto a quello ricorsivo della coreografia) aumenta il carico di messaggistica (comando/risposta per ogni nodo).

La coreografia affida ai servizi coinvolti il compito di gestire autonomamente la logica della saga, reagendo agli eventi generati dagli altri servizi. I principali svantaggi di questo approccio sono la difficoltà di tracciare lo stato della saga (nessun nodo mantiene lo stato dell'intera operazione, ma solo le informazioni sul suo ruolo) e una maggiore complessità nel coordinare le transazioni compensative in caso di fallimento.

Nel complesso, il modello di consistenza offerto dalle saghe passa da avere proprietà ACID (pur valido per le transazioni locali), a proprietà BASE:
\begin{itemize}
  \item Basic Availability: il fatto che ci sia una saga in corso non impedisce a un'altra richiesta di dare inizio a un'altra saga;
  \item Soft state: anche in assenza di nuove richieste esterne, lo stato dei dati potrebbe variare finché tutte le saghe concorrenti non sono terminate;
  \item Eventual consistency: al termine di tutte le saghe concorrenti, i dati saranno in uno stato coerente e letture successive garantiranno gli stessi risultati.
\end{itemize}

\subsection{Disacoppiare client e microservizi: i Gateway API}
Tra i principi di buona progettazione per un'applicazione a microservizi, uno che affronta una varietà di problematiche è il seguente: bisognerebbe limitare le richieste dirette dai client ai microservizi, rendendo l'organizzazione interna quanto più trasparente possibile all'esterno.
Si immagini un'applicazione multipiattaforma in cui ogni categoria di client accede direttamente alle API dei microservizi: la semplice adozione di una scelta simile si ripercuote negativamente su numerosi aspetti dell'applicativo:

\begin{itemize}
	\item una modifica nella separazione delle responsabilità interne avrebbe un impatto elevato anche su tutti i client, necessitando modifiche estensive su tutta la codebase;
	\item un'elevata modularità dell'applicativo sarebbe bilanciata da un incremento delle comunicazioni, dato che ogni client potrebbe arrivare ad aprire una decina di comunicazioni sincrone solo per il caricamento di un'interfaccia grafica;
	\item sarebbe complesso gestire concetti trasversali come autenticazione, SSL, e il dispatch dinamico delle richieste;
	\item la gestione di client di diversi tipi (es. web e mobile) richiede un'elaborazione differente dei dati da presentare all'utente, e tale elaborazione non può essere affidata a dispositivi come quelli mobili, le cui risorse limitate richiedono che il server ottimizzi la comunicazione per raggiungere l'obiettivo con un impegno minimo da parte del client.
\end{itemize}

È facile intuire come, in una molteplicità di casi, la soluzione migliore sia quella di introdurre un tier intermedio, una facciata, che si occupi di accettare le singole richieste da parte dei client e tradurle in una moltitudine di richieste interne che riflettano la struttura corrente dei microservizi, astraendo dal client complessità e conoscenza dell'architettura del server.

\subsubsection{Che cos'è un Gateway API?}
Il pattern dedicato alla risoluzione di questo problema è il cosiddetto \emph{Gateway API}.
Un gateway API è un servizio che costituisce un punto di accesso centralizzato per certi gruppi di microservizi; spesso è chiamato anche Backend For Frontend (BFF) poichè rappresenta la porzione di architettura server progettata per venire incontro alle necessità delle applicazioni client.
Dipendentemente dalla complessità della soluzione, lo strato di logica di facciata può essere anche costituito più gateway API scalabili in maniera indipendente, che si occupino di gestire tipologie accesso ai servizi differenti o per client differenti. Un'architettura di backend potrebbe ad esempio avere un gateway per i client desktop e web, uno per un'app mobile e uno per l'accesso ai servizi di amministrazione del sistema.

I gateway API non sono solo uno strato intermedio d'interfaccia tra client e microservizi, ma un vero e proprio tier a sé che concentra tutte le funzionalità che riguardano l'interazione tra di essi. Nel resto di questa sezione analizzeremo gli scopi principali di un gateway API e gli svantaggi derivati dall'impiego di questo nuovo layer, fornendo delle linee guida su quando adottare o meno tale pattern.

\subsubsection{Feature principali}
A seconda della complessità desiderata, un gateway API può concentrare un insieme più o meno ricco di funzionalità.
La prima, e quella più frequentemente integrata, è il Reverse proxy: il gateway si espone come punto d'accesso centralizzato a un certo set di funzionalità fornite dai microservizi del backend. I client richiedono servizi al server mediante l'endpoint del gateway e questo, agendo da reverse proxy di livello applicazione, smista le richieste (solitamente HTTP) verso i microservizi incaricati di assolvere alla funzionalità d'interesse.
L'utilità principale di questa feature è il totale disaccoppiamento dei client dai microservizi, tuttavia un reverse proxy risulta utile in software che mirano ad una separazione in microservizi, ma partono da un'architettura monolitica: in casi come questo è possibile "nascondere" l'intera architettura dietro gateway API, modificando il routing interno dei proxy via via che le funzionalità si spostano dal monolite legacy a nuovi microservizi, con impatto nullo sul codice dei client.

Un'altra feature che è possibile implementare in un gateway API è l'aggregazione delle richieste, ossia la possibilità di trasformare una richiesta da parte del client all'endpoint del gateway in un insieme di richieste verso più microservizi. Ciò è particolarmente utile quando un client deve caricare una pagina o schermata che mostra informazioni dipendenti da microservizi diversi: il client può inviare un'unica richiesta e il gateway può prendersi carico di effettuare richieste multiple al backend e aggregare le risposte in un'unica risposta per il client.
Bisogna tenere a mente che l'overhead di comunicazione che l'aggregazione si propone di risolvere è significativa per client remoti che si fanno carico dell'elaborazione, per cui spesso non vale la pena introdurre tale servizio per ambienti la cui interfaccia è generata server-side; è il caso ad esempio di ASP.NET Core MVC, il servizio per interfacce web utilizzato per il servizio di reportistica del caso d'uso, descritto nel dettaglio più avanti nel testo.
Pur non essendo garantito che le soluzioni gateway API disponibili consentano l'aggregazione di default, è sempre possibile ottenere tale feature implementando la logica su un microservizio ausiliario che comunichi con il gateway stesso, ottenendo peraltro il massimo livello di personalizzazione.

Il resto delle funzionalità che è consigliabile introdurre in un gateway API rientrano nei cross-cutting concerns, ossia quelle categorie di feature trasversali e di feature che possono riguardare più microservizi e che è pertanto conveniente astrarre in un tier superiore piuttosto che implementare in ognuno dei microservizi. Ricordiamo tra queste i servizi di autenticazione/autorizzazione, che gestiscono l'identità e i permessi degli utenti per garantire la sicurezza; il caching delle risposte per incrementare le prestazioni memorizzando i risultati delle richieste frequenti; i servizi e guarantees per la comunicazione (QoS, retry, circuit breaker...) per l'affidabilità delle interazioni tra i componenti; il load balancing per distribuire il traffico in modo efficiente, prevenendo sovraccarichi; rate limiting per impedire un numero abusivo di richieste al servizio; il throttling, che regola la velocità di elaborazione; un logging centralizzato che raccoglie la diagnostica nel crocevia delle comunicazioni.\cite[44-45]{.NET_Microservices}

\subsubsection{Gli svantaggi di un tier intermedio}
L'introduzione di un gateway API non è una decisione da prendere alla leggera, in quanto l'aggiunta di un ulteriore livello di astrazione e comunicazione comporta inevitabilmente alcuni svantaggi.
Creare uno o più punti di accesso centralizzato significa introdurre dei \emph{single point of failure} nell'architettura: un guasto nel gateway renderebbe inaccessibili tutti i microservizi che espone, anche se questi operano correttamente. Soluzioni a tale problematica includono l'introduzione di sistemi di ridondanza per garantire l'alta disponibilità del servizio, ma ciò comporta un aumento della complessità e dei costi operativi che non sempre vale il guadagno nelle comunicazioni ottenuto con l'uso di gateway.

I gateway sono fortemente accoppiati con i microservizi che espongono, per cui l'introduzione di questo nuovo tier si traduce nella necessità di aggiornare un secondo set di componenti al modificarsi del backend, e persino tecniche avanzate DevOps e CI/CD hanno un limite nella capacità di mitigare gli svantaggi di tale accoppiamento.

Inoltre, questo strato intermedio da cui passano tutte le comunicazioni può facilmente tramutarsi in un \emph{bottleneck} se il sistema non è ben progettato per la scalabilità, sia da un punto di vista delle prestazioni sia per quanto concerne il coordinamento dei team di sviluppo, che come nell'approccio monolitico si ritrovano a dover collaborare su un unico componente che ospita logica da contesti differenti.\cite[47-48]{.NET_Microservices}


\section{L'importanza dei container per le architetture a microservizi}
Potremmo pensare ai microservizi come "pacchetti" preconfezionati di software appartenente a un certo dominio, e ciò ne descriverebbe intuitivamente i vantaggi. Tuttavia, quest'astrazione nasconde una serie di problemi che sono invece cruciali, soprattutto per la distribuzione di applicativi in ambienti server e non solo in architetture a microservizi.
Tali problemi riguardano principalmente l'isolamento, la flessibilità di configurazione e la portabilità delle soluzioni software.Basti pensare al comicamente noto problema del "sul mio computer funziona": un codice teoricamente corretto e funzionante potrebbe risultare inutilizzabile quando esportato su un nuovo calcolatore.
Conflitti con variabili d'ambiente, chiavi di registro o altri elementi del file system; problemi di autorizzazione; assenza o errata configurazione o versione delle dipendenze. Queste sono solo alcune delle innumerevoli e frustranti problematiche che derivano dalla vicendevole contaminazione dell'ambiente in cui gli applicativi software vengono eseguiti.

C'è di più: finora abbiamo ignorato il \emph{come} l'applicativo debba essere esportato su altri calcolatori: anche questa fase costituisce una fonte inesauribile di frustrazioni aleatorie, legate alle differenze tra le architetture dei processori, dei sistemi operativi, del supporto di date funzionalità. Più in generale, sarebbe ideale poter installare l'applicativo come uno \emph{standalone} compreso delle dipendenze necessarie e isolato rispetto allo stato del sistema che lo esegue così da evitare problemi di configurazione. Ci si auspica inoltre di poter mettere facilmente a disposizione l'applicativo su Internet (così da non dover caricare manualmente una versione eseguibile dell'applicativo su ogni server), magari in più versioni e potendo configurare eventuali variabili system-specific del software in maniera semplice e integrata all'avvio.

Tutto questo e di più è possibile all'introduzione di un unico potente concetto: \textbf{i container}, strutture virtuali che consentono di confezionare le applicazioni come pacchetti pronti all'uso.
Costituendo lo standard di fatto per la containerizzazione delle applicazioni, nel prosieguo della trattazione mi riferirò specificatamente all'uso dell'ecosistema \emph{Docker} per descrivere le potenzialità di questa tecnologia.

\subsection{Che cos'è un container?}
Un container è un processo autonomo che include tutto e solo il necessario per eseguire un'applicazione: codice/eseguibili, runtime, dipendenze e variabili d'ambiente. Quando un ambiente software è confezionato in un container, tutto il necessario alla sua corretta esecuzione è mantenuto in uno spazio utente isolato nel sistema operativo host (quello su cui è eseguito), evitando conflitti con altri applicativi e servizi e garantendo che l'applicazione funzioni sempre come previsto, indipendentemente dall'ambiente di esecuzione.
Tali proprietà fanno sì che la containerizzazione sia l'approccio ideale per distribuire facilmente applicazioni software in ambienti distribuiti, in cui l'avvio di un'istanza di software può essere determinato in modo dinamico e automatico da sistemi di orchestrazione: diventa imprescindibile essere in grado di dispiegare più applicativi (anche di più istanze dello stesso) su dispositivi eterogenei e \emph{on the fly}, senza che un eventuale sistema di orchestrazione debba tenere conto di possibili conflitti. Con l'adozione dei container, i server dalle architetture più disparate possono essere considerati come risorse omogenee, delle griglie di "slot per container".

L'unico requisito affinché un server possa usufruire di questa feature "\emph{Build Once, Deploy Everywhere}" è l'installazione di un \emph{Docker Engine}, una piattaforma che contiene:
\begin{itemize}
  \item Docker Daemon (\texttt{dockerd}), servizio che gestisce i container e le altre entità legate al mondo Docker;
  \item Docker API, che consente di interagire con il Daemon tramite richieste REST;
  \item Docker CLI, interfaccia a riga di comando che consente di interagire con il Daemon tramite comandi testuali.
\end{itemize}

\subsubsection{Container contro Virtual Machine}
Non è raro che qualcuno paragoni e confonda il concetto di Container con quello di Virtual Machine, in virtù del fatto che ogni container possiede un proprio spazio utente, con variabili d'ambiente, file system e tool di sistema diversi. In realtà, ogni container \emph{crede} di gestire un proprio sistema operativo, ma la realtà è più complessa di così.
Le macchine virtuali (VM) sono ambienti di esecuzione completi, con un sistema operativo proprio il cui kernel può differire da quello della macchina host grazie a una mappatura tra il sistema virtuale e quello reale.

Al contrario, i container condividono il kernel dell'host (o una VM fornita dall'host, come accade per i container Linux su Windows/MacOS), creando una copia personale solo dello spazio utente che spesso è minimale per garantire agilità nell'operare con essi. Essendo eseguiti direttamente sul sistema host, i container sono notevolmente più leggeri delle VM, somigliando più a dei "processi imbottiti" che a degli ambienti di esecuzione a sé stanti.

\subsection{Ciclo di vita di un container}
Come si fa a distribuire un'applicazione come container?
In realtà, il container in sé rappresenta la singola istanza di un ambiente isolato di esecuzione, mentre l'artefatto che viene distribuito per consentire la creazione di container è detto \emph{immagine}. Per spiegare intuitivamente la relazione tra immagini e container, riassumiamo il flusso da applicazione software a container.

\begin{itemize}
  \item Sviluppo dell'applicazione: il codice sorgente dell'applicazione viene realizzato e testato in un ambiente di sviluppo.
  \item Scrittura del Dockerfile: in base della configurazione desiderata, si compone un file di testo, chiamato \emph{Dockerfile} (privo di estensione), che contiene le informazioni necessarie per costruire l'immagine del container. Tra le altre cose, è obbligatorio specificare la base, un'immagine di partenza a cui aggiungere l'applicativo sviluppato, oltre a dipendenze, variabili e altre configurazioni. Una base può contenere una versione minimale di uno spazio utente, o prevedere delle componenti aggiuntive ottimizzate e pre-installate, come i runtime necessari per l'applicazione. Le immagini di base possono essere caricate dal file system locale o scaricate a \emph{build time} da un registro di immagini online(come Docker Hub).
  \item Build dell'immagine: a partire dal codice sorgente dell'applicazione e dal Dockerfile, viene generata l'immagine, che rappresenta l'insieme spazio utente + applicazione come definito dalle istruzioni del Dockerfile.
  \item Distribuzione dell'immagine: l'immagine viene caricata su un registro di immagini (pubblico o privato) da cui può essere scaricata per creare nuovi container.
  \item Esecuzione del container: scaricando un'immagine, è possibile creare un nuovo container come "istanza" di essa, eventualmente configurabile con variabili e parametri contestuali tramite Docker CLI.
\end{itemize}

\subsection{Persistenza e connettività dei container}
La piattaforma Docker non si limita a gestire ambienti di esecuzione, ma offre anche varie funzionalità per consentire ai container di salvare i dati in modo persistente e di comunicare tra loro, con altri processi e con l'esterno.

\subsubsection{Gestione dei dati}
Il file system di un container Docker è effimero per definizione: quando il container viene eliminato, anche i dati al suo interno vanno perduti. Per questo motivo, Docker offre alcune alternative per preservare i dati sull'host.

Il \textbf{Bind mount} consiste nel mappare una cartella del filesystem dell'host su una cartella del container: in questo modo, i dati permangono anche con l'eliminazione del container. Questa soluzione porta un evidente svantaggio di portabilità, in quanto bisogna garantire che il ramo specificato sia sempre presente sull'host su cui viene eseguito il container: ciò causa problemi per path relativi e interventi inattesi sul ramo da parte dell'host.

Un \textbf{Named Volume} è uno storage dedicato sull'host e gestito da Docker indipendentemente da esso. Questa soluzione offre una maggiore portabilità ed è più automatica; per contro, accedere allo storage dall'host è meno intuitivo che navigare verso una cartella nota a priori (come per un bind mount).

Una soluzione non realmente persistente, ma utile in alcuni scenari, è il \textbf{tmpfs mount}, che consente di mappare una porzione di memoria RAM dell'host su una cartella del container. I dati memorizzati in un tmpfs mount sono volatili, ma l'accesso è molto più veloce rispetto a un volume su disco.

\subsubsection{Connettività in rete}
Nonostante il pregio principale dei container Docker sia il loro isolamento, ciò non significa che non dovrebbero poter comunicare tra loro e con l'esterno. Esistono due modi integrati in Docker per fornire connettività ai container.

Una rete \texttt{bridge} è una rete virtuale privata che consente ai container connessi di comunicare tra loro grazie ad alias e un DNS interno. La comunicazione in uscita con l'esterno è consentita di default in quanto un bridge fa NAT (\emph{Network Address Translation}) verso l'host, mentre per abilitare una comunicazione bidirezionale è necessario utilizzare il meccanismo di \emph{port forwarding} fornito da Docker, al fine di mappare porte dei container a porte dell'host.

Una rete \texttt{host} invece consente ai container connessi di condividere lo stesso stack di rete dell'host, il che significa che possono comunicare tra loro e con l'esterno utilizzando l'IP dell'host stesso. Questa soluzione garantisce performance elevate per la comunicazione con l'esterno, ma sacrifica l'isolamento di rete tra i container e l'host.

È possibile inoltre definire delle reti personalizzate, di tipo \texttt{bridge} per personalizzare quali container possono comunicare tra loro, oppure con una configurazione \texttt{macvlan} che consente a ogni container di ottenere un indirizzo MAC virtuale e un indirizzo IP dedicato su una rete LAN. L'introduzione di sistemi di orchestrazione come \emph{Docker Swarm} consente di definire anche reti cosiddette \texttt{overlay}, che consentono a container su host diversi di comunicare tra loro attraverso una rete virtuale distribuita.

\subsection{Coordinare container multipli: da Docker Compose all'orchestrazione}
Come spesso succede con le interfacce testuali potenti, i comandi Docker CLI rischiano di essere verbosi e poco chiari all'aumentare del numero di container da gestire. Fortunatamente, lo stesso ecosistema Docker risolve il problema fornendo metodi per dirigere un numero maggiore di container mediante l'automatizzazione della composizione di comandi Docker CLI che sarebbero altrimenti complessi e ripetitivi.
Il tool che racchiude tali metodi è \texttt{Docker Compose}, un livello di astrazione sopra Docker che consente di coordinare uno stack di container correlati mediante una sintassi dichiarativa semplice su file \emph{YAML} (\texttt{docker-compose.yml}).

Tuttavia, quando si tratta di gestire container in produzione, è necessario adottare approcci più avanzati e scalabili, che consentano di coordinare un numero variabile di container su architetture distribuite. Gli strumenti che consentono tali operazioni sono detti \emph{orchestratori}.

Come spiegato chiaramente sul sito di \emph{Red Hat}: "L'orchestrazione dei container è il processo che permette di automatizzare il deployment, la gestione, la scalabilità e il networking dei container attraverso l'intero ciclo di vita, consentendo di distribuire il software in modo uniforme in molto ambienti diversi e su larga scala." \cite{RedHat_Orchestration}

Senza entrare nel dettaglio, un \emph{orchestrator} si occupa di avviare, fermare, monitorare e replicare i container su diversi nodi di un cluster, garantendo alta disponibilità, bilanciamento del carico, gestione dei guasti e aggiornamenti senza interruzione del servizio. Gli orchestratori come \emph{Kubernetes}, \emph{Docker Swarm} e \emph{Apache Mesos} consentono di coordinare in modo efficiente l'esecuzione di applicazioni composte da sciami di microservizi, semplificando la gestione operativa e migliorando la resilienza dei sistemi distribuiti.
I sistemi di orchestrazione sono centrali in contesti che adottano filosofie come \emph{DevOps} e \emph{CI/CD} (\emph{Continuous Integration/Continuous Deployment}), in quanto consentono di automatizzare il rilascio di nuove versioni del software e la gestione delle infrastrutture in maniera efficiente e affidabile.

\subsection{Microservizi su container: un connubio spontaneo}
Nello sviluppo di architetture a microservizi, l’adozione di container si configura come una scelta quasi naturale, poiché risponde in modo intrinseco alle esigenze di isolamento, portabilità e scalabilità che caratterizzano questo paradigma. I microservizi, essendo entità autonome e indipendenti, traggono beneficio da un ambiente di esecuzione che garantisca un confine netto rispetto alle dipendenze esterne, preservando al contempo la leggerezza e la rapidità di distribuzione. In aggiunta, la granularità con cui i container possono essere orchestrati (tipicamente tramite gli strumenti sopra accennati) si allinea perfettamente al modello dei microservizi, permettendo di scalare selettivamente solo le componenti che presentano colli di bottiglia, ottimizzando così l’utilizzo delle risorse e garantendo elevata resilienza. In tale prospettiva, i container non soltanto uno strumento per aumentare l'efficienza del deploy, ma diviene un abilitatore architetturale che incarna in maniera pragmatica i principi cardine della filosofia a microservizi.
Osserviamo nel dettaglio come il deployment in container risponda a un numero nutrito di esigenze specifiche attese dall'adozione di un sistema a microservizi.
In primo luogo, la portabilità: un microservizio incapsulato in un container può essere eseguito in maniera uniforme su differenti ambienti (dallo sviluppo locale al cluster in cloud), senza dipendere dalle specifiche del sistema operativo o dalle librerie presenti sull’host. In secondo luogo, l’isolamento: ogni microservizio opera all’interno di un contesto dedicato che ne delimita le risorse, le dipendenze e la visibilità di rete, prevenendo conflitti e garantendo maggiore robustezza del sistema complessivo. Altro requisito dei microservizi favorito dalla containerizzazione è la scalabilità dinamica: i container possono essere replicati e gestiti in maniera elastica, consentendo di rispondere con tempestività a variazioni nel carico di lavoro, spesso attraverso gli orchestratori che automatizzano il bilanciamento e il failover. Ancora, la rapidità di provisioning e rilascio: le immagini Docker, essendo versionate e distribuite in registry ottimizzati per il loro storage, consentono un ciclo di Continuous Integration/Continuous Delivery estremamente snello, riducendo i tempi di rilascio e facilitando rollback in caso di regressioni. Infine, la leggerezza e l’efficienza delle risorse: a differenza delle macchine virtuali, i container hanno un footprint ridotto e un avvio pressoché immediato, rendendo più agevole agli elaboratori che ospitano tali ambienti la gestione di un elevato numero di istanze eterogenee. In tal modo, il paradigma dei microservizi trova nel container il proprio substrato tecnico privilegiato, capace di tradurre i principi di indipendenza, modularità e resilienza in pratiche operative concrete ed efficaci.


\section{Esporre servizi sul web}

Quando si progetta un'architettura a microservizi, uno dei principali fattori che impattano le performance, la manutenibilità e la scalabilità è la scelta di come i microservizi espongono e fruiscono delle funzionalità di cui si fanno carico. In altre parole, bisogna definire l'\emph{IPC} (Inter-Process Comunication), ossia la struttura con cui i microservizi consentono agli altri elementi del sistema e ai client di interrogare le funzioni cui si propongono di assolvere.
Ci sono varie categorie di protocolli e standard ben stabiliti, ognuno con i suoi vantaggi e svantaggi e i suoi ambiti applicativi: nel contesto dei servizi web distribuiti, ambito dominante per le architetture a microservizi, si distinguono quattro tecnologie che hanno fatto la storia. Dopo una panoramica sulle tipologie di comunicazione su cui un'API può basarsi, analizzeremo le caratteristiche principali di tali tecnologie e quali utilizzare a seconda dei parametri legati al contesto applicativo specifico.

\section{Tipologie di comunicazione}

Una prima grande distinzione nelle modalità in cui i microservizi possono comunicare è in base al tempo.
Una comunicazione sincrona prevede che chi avvia la comunicazione rimanga in attesa bloccante di un riscontro: per questo motivo, il ricevente dovrebbe essere in grado di rispondere in tempi brevi ed eseguire in una configurazione ad \textbf{alta disponibilità}. Se il destinatario non è attivo o comunque in grado di recepire il messaggio, l'informazione viene persa. Protocolli che supportano comunicazione sincrona sono ad esempio \textbf{HTTP} e \textbf{gRPC}.
Una comunicazione asincrona consiste in mittenti che inviano messaggi senza attendere una risposta, eventualmente gestendo l'esito \textbf{quando arriva la risposta, non quando avviene la richiesta}: i destinatari, se previsto, rispondono quando sono disponibili e ritengono opportuno farlo, permettendo di configurare una comunicazione \textbf{flessibile} e aprendo le porte alle infrastrutture event-driven. I protocolli cruciali in quest'ambito sono quelli basati su messaggi, come AMQP.

Una seconda distinzione riguarda il formato delle informazioni scambiate: si distinguono infatti comunicazioni in formati text-based, come JSON o XML, che prediligono la chiarezza dei dati alla performance, e in formati binari, come il Protocol Buffer, che non temono l'aggiunta di un middleware di codifica al fine di ottenere una comunicazione \emph{fulminea ed efficiente}.

Un'altro parametro centrale nella scelta nello stile di comunicazione da adottare è la molteplicità dei destinatari: in una comunicazione punto-punto (o uno-a-uno), ogni messaggio inviato da un microservizio è inteso per uno e un solo altro microservizio; viceversa, in una comunicazione uno-a-molti, il mittente invia il messaggio a più di un destinatario (non è raro che sia ignoto al mittente chi ascolta dall'altra parte) e solitamente non attende una risposta.

L'ultimo elemento fondamentale che determina lo stile di comunicazione è la tipologia di contenuto dei messaggi scambiati. In realtà questa distinzione non categorizza i protocolli, che possono infatti usare più di contenuto: è una distinzione che riguarda la singola comunicazione, e la prevalenza di una categoria su un'altra delinea la filosofia adottata dall'IPC scelta.
Il messaggi possono essere:

\begin{itemize}
	\item \emph{informazioni}, se il loro contenuto sono esclusivamente dati
	\item \emph{comandi}, se l'intento è quello di avviare una certa procedura in uno o più microservizi in ascolto
	\item \emph{eventi}, se servono a comunicare ai microservizi in ascolto che lo stato del sistema si è evoluto, così che questi possano reagire in modo consono.
\end{itemize}


Volendoci concentrare, per ragioni di brevità, sulle combinazioni di caratteristiche che dominano nel contesto dell'IPC tra microservizi, identifichiamo una prevalenza di comunicazioni \emph{sincrone} di tipo \emph{Request/Response} (di conseguenza, tendenzialmente uno a uno), oppure \emph{asincrone} di tipo \emph{Publisher/Subscriber} (uno a molti) o \emph{Message Queue} (uno a uno).

Per l'IPC sincrona, le soluzioni che costituiscono lo standard di fatto nelle architetture a microservizi sono REST, RPC e GraphQL.

\emph{REST} (\emph{\textbf{RE}presentational \textbf{S}tate \textbf{T}ransfer}) è una filosofia architetturale per l'implementazione di API basate su HTTP acclamata per la facilità di utilizzo e la disponibilità del servizio (è sufficiente HTTP 1.1). Risulta particolarmente vantaggiosa quando il servizio offerto è orientato alle \emph{risorse} piuttosto che alle azioni.

Infatti, tutti i dati scambiati in una comunicazione \emph{RESTful} (aggettivo che descrive un'API che abbraccia i principi della comunicazione REST) sono considerati risorse, la cui manipolazione avviene concordemente al \emph{verbo HTTP} (come GET, PUT, POST e DELETE) associato alla richiesta.
In buona sostanza, il client di una API RESTful può accedere a determinati URI del dominio a cui è esposto il server e scambiare risorse con esso secondo il tipo di verbo HTTP utilizzato. Ad esempio, una GET all'endpoint /products/\{prodId\} potrebbe essere fatta da un client che vuole accedere a un dato prodotto, mentre un POST a /products consente di aggiungerne uno nuovo. È evidente come, per servizi orientati alle risorse, l'utilizzo di REST risulti naturale ed intuitivo, senza dover apprendere schemi complessi diversi da una documentazione che descriva le chiamate possibili ed eventuali parametri (peraltro, sono disponibili numerosi strumenti che consentono di generare automaticamente tale documentazione, ad esempio per il caso di studio nel capitolo 5 il sottoscritto ha usufruito di Swagger per generare un'interfaccia intuitiva e reattiva che descriva efficacemente le chiamate possibili e la loro struttura).

In realtà, i vincoli da rispettare affinché un'API sia considerata pienamente RESTful sono più stringenti. Accenniamoli brevemente:

\begin{enumerate}
  \item \textbf{Architettura Client-Server}: Il client è responsabile della user interface e dello stato dell'applicazione, mentre il server gestisce le risorse e il loro stato.
	\item \textbf{Stateless}: Ogni richiesta dal client al server deve contenere tutte le informazioni necessarie per essere compresa. Il server non deve conservare alcuna informazione di sessione relativa al client tra una richiesta e l'altra. Se un'autenticazione è necessaria, ogni richiesta deve includere le credenziali o un token di autenticazione.
	\item \textbf{Cacheability}: Le risposte del server devono essere esplicitamente o implicitamente definite come memorizzabili nella cache o meno. Le risposte memorizzabili nella cache riducono la latenza e il carico sul server, migliorando le prestazioni dell'API. Se una risorsa viene contrassegnata come "cacheable", il client può riutilizzare una risposta precedentemente ricevuta per una richiesta successiva, senza dover contattare nuovamente il server.
	\item \textbf{Interfaccia Uniforme}:
	\begin{itemize}
		\item \textbf{Identificazione delle risorse}: Le singole risorse sono identificate da URI.
		\item \textbf{Manipolazione mediante rappresentazioni}: Quando un client vuole manipolare una risorsa, ne richiede una rappresentazione (in un formato come XML o JSON), la modifica e invia una richiesta al server con la rappresentazione modificata per istruirlo sulle modifiche da apportare.
		\item \textbf{Messaggi auto-descrittivi}: Ogni messaggio (richiesta e risposta) deve includere informazioni sufficienti per essere elaborato.
		\item \textbf{Hypermedia as the Engine of Application State (HATEOAS)}: Il client non deve conoscere a priori l'URI di ogni risorsa, ma è compito del server arricchire le proprie risposte con link a risorse correlate che potrebbero essere d'interesse.
	\end{itemize}
	\item \textbf{Sistema a strati} La richiesta del client può essere gestita da un intermediario in vece del server finale, senza che il primo ne sia consapevole.
	\item \textbf{Code on Demand}: Facoltativo. Al server è concesso di estendere la funzionalità del client inviando codice eseguibile (ad esempio, JavaScript o applet).
\end{enumerate}

Elenchiamo di seguito alcuni svantaggi nell'utilizzo dell'approccio REST, al fine di guidare una scelta consapevole sulla tecnica da adottare per un caso d'uso specifico:

\begin{itemize}
	\item REST è pensato esclusivamente per una comunicazione sincrona \emph{uno-a-uno} e \emph{Request/Response}, risultando inutilizzabile per qualsiasi altro approccio;
	\item Risulta meno intuitivo, e anzi forzato, usare REST contesti applicativi in cui le operazioni da eseguire non sono prevalentemente di tipo \emph{CRUD} (\emph{\textbf{C}reate, \textbf{R}ead, \textbf{U}pdate, \textbf{D}elete}), ossia operazioni che creano, leggono, aggiornano o eliminano risorse. In questi casi, un paradigma orientato alle azioni (come RPC) risulta più naturale e performante;
	\item Basandosi esclusivamente su HTTP, REST non fornisce un modo semplice per ottenere risorse multiple con una sola richiesta: l'approccio atteso sarebbe una chiamata a sé stante per ogni risorsa, il che si traduce in estesi \emph{overhead} che sarebbero altrimenti evitabili;
	\item L'assenza di un broker per la gestione della comunicazione rende REST semplice e snello, ma HTTP da solo non offre tecniche di message buffering, per cui tale approccio è conveniente solo in presenza di un server ad alta disponibilità, che garantisca di rispondere al client sempre e subito;
	\item Poiché gli endpoint sono esposti tramite URL, REST risulta poco flessibile in un contesto in cui l'API potrebbe evolvere o migrare. Per garantire una maggiore flessibilità su comunicazione REST, sarebbe necessario introdurre un meccanismo di \emph{discovery} aggiuntivo che consenta ai client di localizzare dinamicamente gli endpoint dell'API.
\end{itemize}

Nel complesso, le caratteristiche descritte delineano REST come uno stile che predilige la semplicità e la chiarezza al di sopra di performance e funzionalità.
Una soluzione talvolta proposta come alternativa più strutturata a REST è \emph{GraphQL}.
Si tratta di un linguaggio d'interrogazione per API che, al costo dell'introduzione di un nodo intermediario e dell'utilizzo di librerie aggiuntive sia lato client che lato server, vanta un sistema di richieste mirate che, previa definizione di uno schema delle risorse, consente al client di ottenere tutte e sole le informazioni necessarie con un'unica richiesta. Tale caratteristica sfida REST in una delle sue principali carenze: la granularità delle risorse è definita dall'API, per cui una distribuzione delle risorse fondamentale per alcune operazioni a grana fine può invece risultare tediosa e mal performante per operazioni su più ampia scala. Inoltre, l'approccio REST di associare ad ogni URI (e dunque ad ogni richiesta) una sola risorsa, impone al client di dover definire internamente algoritmi per il retrieval di informazioni strutturate che si traducono inevitabilmente in sequenza anche estese di richieste HTTP consecutive.

Altre funzionalità che elevano le potenzialità d'uso di GraphQL sono le migration, ossia la possibilità di specificare, direttamente a partire dallo schema delle risorse, delle modifiche sui dati da richiedere al server, e le subscription, ossia la posssibilità di "iscriversi" a date risorse per essere notificati dei loro aggiornamenti.

A seconda della complessità dell'API, dei costi, delle risorse computazionali a disposizione e della chiarezza target, GraphQL può o meno essere considerata un'alternativa preferibile a REST; in ogni caso, quest'ultima rimane la soluzione di gran lunga più adottata nel contesto delle web API per microservizi.

L'altro capostipite dell'IPC sincrona è il paradigma \emph{RPC} (\emph{\textbf{R}emote \textbf{P}rocedure \textbf{C}all}). La caratteristica fondamentale che distanzia REST e RPC è che quest'ultimo standard è pensato per elaborare \emph{azioni}, non risorse.
Un'immediata conseguenza di una filosofia orientata alle azioni è che l'intento delle chiamate, qui centro del paradigma, non viene più scandito dai verbi HTTP. Lo standard prevede infatti di limitarsi a richieste GET per le azioni read-only e richieste PUT per le operazioni che forniscono o alterano informazioni.
Nella realizzazione di interfacce con maggior riguardo alle operazioni, RCP rappresenta lo stile più autodescrittivo e di facile comprensione, ed è caratterizzato da performance generalmente più elevate, poiché il payload delle richieste sarà solitamente circoscritto alle informazioni strettamente legate all'azione intrapresa, a differenza di REST che prevede quantità non indifferenti di risorse trasmesse ma inutilizzate.
I principali drawback di questo stile di progettazione di API derivano dalla modalità di esposizione delle funzionalità: senza vincoli sull'organizzazione e la presentazione delle azioni, la disponibilità di una documentazione ben realizzata diviene imprescindibile, mentre con REST risultava intuitivo comprendere quali verbi siano applicabili a una data risorsa; al contempo, il fatto che ogni azione sia associata a un URL diverso pone il rischio non indifferente di una crescita sregolata degli endpoint all'aumentare delle funzionalità fornite. Un'implementazione che si è stabilita sopra le altre è \textbf{gRPC}, che si distacca ulteriormente dai principi di REST proponendo un sistema di invocazioni basate su \textbf{ProtoBuffer}, tecnologia che permette una formattazione binary-based dei messaggi per garantire efficienza elevata rispetto alle chiamate ripetute e ingombranti di REST.
gRPC si propone come soluzione cross-language per ottenere chiamate efficienti e fortemente tipizzate nonstante l'approccio language-neutral della comunicazione, ottenuto grazie a una sintassi di definizione delle procedure che costituisce un accordo type-safe tra interfacce basate su framework e linguaggi distinti.

\section{Evoluzione delle applicazioni web}
%comunicazione asincrona?
%mpa, spa, ajax, sse, websockets, js vs webassembly

\section{}

%sicurezza e autenticazione: oidc, jwt, antiforgery, cookies, cors
