\chapter{Conclusioni e sviluppi futuri}
La soluzione prototipale realizzata si configura come un punto di partenza architetturalmente solido per un sistema che mira ad un deploy distribuito, e vanta un'attenzione alla modularità, alla configurabilità e all'accoppiamento che consentiranno, senza particolari sforzi, di abilitare il supporto a protocolli di sicurezza basati su certificato (SSL/TSL) e sostituire implementazioni e strumenti "di development", quali i database SQLite o l'orchestrazione con Docker Compose, con soluzioni più robuste e resilienti, pronte per l'ambiente di produzione.

Durante la fase di implementazione del software il sottoscritto ha avuto modo di riflettere, assumendo il desiderio di preservare lo stato corrente della libreria di reportistica al cuore del progetto, sulla necessità di produrre un'API di alto livello che consenta di configurare il tipo e i contenuti dei documenti generati mediante un sistema di "\emph{plugin}", in un'ottica di evoluzione futura del progetto. Ciò verrebbe incontro all'ideale, promosso nel corso di questa tesi, di rendere il servizio altamente e facilmente configurabile, a un livello più profondo rispetto all'esoscheletro di servizi web sviluppato: sarebbe così possibile realizzare entità software che determinino i criteri di formattazione e popolazione dei documenti in maniera personalizzata per le esigenze delle aziende e delle varie categorie di operatori, e "caricare" dinamicamente il motore di generazione con tali entità.

Non si nasconde al lettore una certa difficoltà riscontrata da chi scrive nel configurare correttamente framework così potenti e intrecciati fra loro: le nozioni e gli strumenti descritti in questo testo sono evidentemente più complessi e con sfaccettature maggiori rispetto ai cenni che sono stati possibili in questa sede, e assimilare l'intero carico di informazioni nell'arco temporale dell'esperienza di tirocinio curriculare ha dettato un apprendimento basato sugli errori di implementazione e non senza parecchia resilienza.
Ciò conferma come La realizzazione di questo progetto si sia rivelata un'esperienza formativa anche in un ambito più trasversale, insegnando al sottoscritto quanto si possa realizzare partendo da un obiettivo e desiderio di apprendere, e quanto tale approccio sia tuttavia insufficiente quando il traguardo è una soluzione ingegneristica ben strutturata, specie in ambito informatico.

Nel complesso, la scelta di un tirocinio formativo che esulasse dai precetti appresi in ambito accademico si è rivelata stimolante, completando ulteriormente il ventaglio di conoscenze panoramiche che si addice a un dottore in Ingegneria Informatica e stimolando ancor di più il desiderio di comprendere i paradigmi e le filosofie alla base dei complessi sistemi informatici che oggi sono per noi così indispensabili.